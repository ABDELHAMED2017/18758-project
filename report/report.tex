\documentclass[11pt]{scrartcl}

\usepackage{hyperref}
\hypersetup{colorlinks=true,urlcolor=blue}
\urlstyle{same}

\begin{document}

\title{Final Project Report}
\subtitle{18-758 Wireless Communications}
\author{Michael Nye (mnye)}
\date{}
\maketitle



\section*{System Design}

\subsection*{Modulation}
For my final project, I designed a system fairly similar to what was shown in
lectures. For coding, I used a rate-1/2 convolution code with 4 states. After
codig, the symbol bits are interleaved by a factor of 132 (chosen to be a large
integral factor of my packet size). Again, testing of my system has demonstrated
that this coding scheme sufficiently decouples error events and allows robust
correction given my channel.

I use an 8-PSK modulation scheme. My choice of eight points was
motivated by empirical measurements of the channel; I used the highest order
constellation I could support without the noise causing symbol errors. I then
used a hamming window pulse. The hamming window was chosen as it has better
bandwidth properties than other FIR windows, but was easier to implement than
a multisymbol pulse such as a raised cosine rolloff pulse. Testing showed that
this pulse was sufficient to meet the project specifications.

Finally, I prepend my message with a single pilot sequence to facilitate
equalization and timing synchronization. The sequence is created by generating
a bit sequence. This sequence is a De Bruijn sequence, which creates a large
variety in symbol ordering to create a unique sequence. This sequence is then
modulated using a wide BPSK for easy detection.

Constant factors were chosen to be as large as possible while still meeting 
channel requirements and message size. Since I was fixed to rate-1/2 coding
and 3 bits per symbol, this meant to fit my entire message (3036 pixels) in
the allowed space, I was able to use at most 4 samples per symbol. I then
expanded my pilot sequence to fill most of the remaining space.


\subsection*{Demodulation}
My demodulator largely follows the reverse of my modulator. First off, I perform
carrier recovery. To do so, I compute a DTFT on my received signal over a range
of only low frequencies. I find the maximum amplitude frequency, and then
divide by a complex sinusoid of the same frequency.

This is followed by timing recovery and equalization. I find the correlation
between my pilot signal and the recovered signal. The maximum lag is the start
of my pilot sequence. I can then extract the pilot sequence, and detect the
pilot symbols. The channel equalization constant is then found by comparing
the detected symbols to the known pilot symbols. Finally, the received signal
is divided by this same factor to equalize.

At this point, I can window out only my message. I match filter the message,
then sampling the result and perform hard detection of the coded bits. After
deinterleaving these detected bits, I find the minimum error path through my
coding trellis to correct any bit errors in the coded bits. Finally, the
corrected coded bits are decoded into my message bits and returned.




\section*{Analysis}

My system largely follows the principles demonstrated in lectures. There
are three main differences I chose to make to simplify the system.

The first difference was my choice of pilot. I chose to use the same pilot
message for carrier recovery, timing recovery and equalization. My testing
found that this didn't cause any degradation of performance, and allowed me
to minimize the size of my header.

The second difference was my choice of window. While raised cosine rolloff
windows are clearly superior to hamming windows in terms of bandwidth properties,
I found it difficult to correctly design the window to prevent ISI. A hamming
window was much simpler to implement and performed admirably for my purposes.

The final difference was in how I performed carrier recovery. While we were
encouraged to use a BPSK pilot sequence, I found this didn't provide sufficient
resolution to find the carrier offset frequency. I instead just used the entire
received signal, and bandlimited the region I search for the peak in the
frequency domain. This proved surprisingly effective, and so I stuck with it.

The other difficulties I ran into were simply implementation details. Until this
project, I had never used the MATLAB \verb|'| operator to transpose complex data.
I learned that this was a hermitian transpose, but it lingered in dark corners of
my code base, and took quite a while to track down the errors. I also spent a lot
of time determining exact offsets for my sampling times. This complexity was 
simply an artifact of using convolution with MATLAB vectors, and the resulting
offsets it introduces into my data.


\end{document}
